%Quelle: https://tex.stackexchange.com/questions/515525/how-to-make-a-custom-requirement-card-like-voleres
\section*{Anforderungen an das Administrationssystem}
\subsection*{Funktionale Anforderungen}
\begin{myreq}
  \threeinline
    {\reqno 9}
    {\reqtype Funktional}
    {\reqevent Manipulieren von Aufgaben}
  \reqdesc Es soll eine grafische Oberfläche zur Manipulation von Aufgaben geben.
  \reqrat Als Kunde möchte ich alle sieben Aufgabentypen anlegen, bearbeiten, speichern und löschen können.
  \reqorig Kunde
  \reqfit Erfüllt, wenn man für jeden Aufgabentyp alle oben genannten Operationen ausführen kann.
  \twoinline
    {\reqsatis 2}
    {\reqdissat 4}
  \twoinline
  {\reqdep -}
  {\reqconf -}
  \reqmater -
  \reqhist 14.11.2022 - erstellt, 19.11.2022 - überarbeitet
\end{myreq}

\begin{myreq}
  \threeinline
    {\reqno 10}
    {\reqtype Funktional}
    {\reqevent Manipulieren von Accounts }
  \reqdesc Der Administrator soll die Möglichkeit haben, neue Accounts anzulegen und bestehende Accounts zu löschen.
  \reqrat Als Administrator möchte ich für neue Studierende Accounts erstellen können und bestehende Accounts löschen können, wenn Studierende die Universität verlassen.
  \reqorig Kunde 
  \reqfit Erfüllt, wenn der Administrator Accounts erstellen und löschen kann.
  \twoinline
    {\reqsatis 4}
    {\reqdissat 3}
  \twoinline
  {\reqdep -}
  {\reqconf -}
  \reqmater -
  \reqhist 18.11.2022 - erstellt, 19.11.2022 - überarbeitet
\end{myreq}

\subsection*{Nichtfunktionale Anforderungen}

\begin{myreq}
  \twoinline
    {\reqno 11}
    {\reqtype Nicht Funktional}
    {\reqevent Intuitive Darstellung}
  \reqdesc Die Benutzeroberfläche soll möglichst intuitiv angeordnet werden, um einen schnellen Einstieg in das Programm zu gewährleisten.
  \reqrat Als Kunde möchte ich mich auf das Lehren konzentrieren und nicht gegen ein Programm kämpfen.
  \reqorig Kunde
  \reqfit Ein Test mit mehreren Personen wurde durchgeführt. Dabei wurden alle täglichen Use-Cases abgedeckt und validiert, dass bei der Navigation und Benutzung keine größeren Probleme aufgetreten sind.
  \twoinline
    {\reqsatis 1}
    {\reqdissat 4}
  \twoinline
  {\reqdep -}
  {\reqconf -}
  \reqmater -
  \reqhist 14.11.2022 - erstellt, 19.11.2022 - überarbeitet
\end{myreq}